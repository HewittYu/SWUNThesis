\documentclass[a4paper,twoside]{ctexrep} % 使用 CTEX 模板
%\usepackage{template/swunthesis-master-professional} %学硕注释掉这一行
\usepackage{template/swunthesis-master} %专硕注释掉这一行
%===============================================
\usepackage{ctex}
\usepackage{fancyhdr}
\usepackage{float} %强制固定图片位置
\usepackage{enumitem}
\usepackage{bicaption}
\usepackage{graphicx}
\usepackage{caption}
\usepackage{titletoc}
\usepackage{tabularx}
\usepackage{lipsum} % 使用时删除

\title{基于\LaTeX 的西南民族大学(专业)硕士学位论文排版格式}
\entitle{The Typesetting Format of (Professional) Postgraduate Dissertation of Southwest Minzu University based on \LaTeX }

\author{于博洋}
\enauthor{Boyang Yu}

\adviser{蔡英}
\coadviser{XXX} % 校外导师(only专业硕士学位)


\college{电子信息学院}
\encollege{College of Electronics and Information}

\major{人工智能} % only学术硕士学位
\research{电子信息}  %学科门类 or 类别(领域)
\resdirection{计算机视觉} % 研究方向
\grade{2020级}

\findate{2024年x月x日} %完成日期
\date{\today} %定稿日期、签署授权书日期

\usepackage{fancyhdr}
\DeclareCaptionLabelFormat{english}{#1~\textbf{#2}}

%设置目录样式,不建议修改
%%%%%%%%%%%%%%%%%%%%%%%%%%%%%%%%%%%%%%%%%%%%%%%%%%%%%%%%%%%%%%%%%%%
\titlecontents{chapter}[3em]{\zihao{4}\songti}{\contentslabel{2.5em}\hspace{1em}}{}{\titlerule*[8pt]{.}\contentspage}
% 设置节目录样式为小四号宋体
\titlecontents{section}[5em]{\zihao{-4}\bfseries \songti}{\contentslabel{2.5em}}{}{\titlerule*[8pt]{.}\contentspage} 
% 设置小节目录样式为小四号宋体
\titlecontents{subsection}[6.5em]{\zihao{-4}\songti}{\contentslabel{3em}}{}{\titlerule*[8pt]{.}\contentspage}
%%%%%%%%%%%%%%%%%%%%%%%%%%%%%%%%%%%%%%%%%%%%%%%%%%%%%%%%%%%%%%%%%%%
\begin{document}
\makecover

\begin{study}

\end{study}


% 摘要
\begin{abstract}{关键词1;关键词2;关键词3}  %关键词位置
这是一个中文摘要示例。
\end{abstract}

\begin{abstractEng}{Keywords 1; Keywords 2; Keywords 3}
This is an example of Abrtract in English.
\end{abstractEng}


\begin{tocc}

\end{tocc}

\begin{foreword}
当撰写论文的前言时,应该包括一系列内容以引导读者对论文有清晰的认识。首先,前言应该简要介绍研究领域的背景和当前的研究现状,明确指出该研究所处的学术位置和价值所在,以及解决的具体问题或探索的内容。其次,要明确阐述论文的研究目的和意义,即研究的目标、动机和预期成果,让读者了解研究的价值和意义所在。接着,论文前言还应描述研究所采用的方法和技术路线,说明研究的思路和实施过程,确保读者对研究过程有清晰的认识。此外,还要对论文的结构安排进行简要介绍,包括各章节的主要内容和组织结构,以便读者更好地理解论文的整体架构和内容安排。最后,在前言中还可以包括对研究过程中获得支持和帮助的感谢之词,向给予帮助和支持的个人、机构表示诚挚的感谢。综上所述,论文的前言应该清晰明了地阐述研究背景、目的和意义,描述研究方法和技术路线,介绍论文的结构安排,同时表达对支持和帮助的感谢之情,为读者提供良好的阅读导引和理解框架。
\end{foreword}

\zihao{-4}
%按照毕业论文格式,重新设置页码(建议不要删除)
\cfoot{\thepage} % 页码居中显示
\setcounter{page}{1}
\pagestyle{fancy}

% 正文示例,按下方样式引入新的章节
\chapter{绪~~论}
\section{示例章节1}
\zihao{-4}
这是一个示例章节.

\lipsum[1-3] % 正式使用模版时删除
\footnote{这里是脚注内容。}
\section{国内外研究现状}
这是一个示例引用。\textsuperscript{\cite{dongshihai2004}}

\subsection{正文示例}
\zihao{-4}
以下是一个正文内容示例。

\begin{enumerate}
	\item 这是一个有序号排列。
	\item 同上。
\end{enumerate}

\begin{itemize}
	\item 这是一个无序号排列。
	\item 同上。
	
\end{itemize}


\begin{figure}[htbp!]
	\centering
	\includegraphics[scale=0.5]{figures/logo.jpg}
	\caption{这是图片的中文标题}
    \captionsetup{labelformat=english}
    \caption*{\bfseries Figure~1-1~This is a sample of image} %需要手动修改
    \label{fig1}
\end{figure}
\begin{table}[H]
	\centering
	\caption{这是一个表格示例}
    \captionsetup{labelformat=english}
    \caption*{\bfseries Table~1-1~This is a sample of table.}
	\begin{tabularx}{\textwidth}{>{\centering\arraybackslash}X>{\centering\arraybackslash}X>{\centering\arraybackslash}X}
			\toprule
			名称 & 名称 & 名称 \\ \midrule
			内容 & 4, 5, 6 & 123 \\
			内容 & 4, 5, 6 & 123 \\ \bottomrule
	\end{tabularx}
\end{table}
\chapter{示例章节}
\lipsum[3] % 正式使用模版时删除
\footnote{这里是脚注内容2。}

\section{示例Section-2}
\begin{figure}[htbp!]
	\centering
	\includegraphics[scale=0.1]{example/template/title.jpg}
	\caption{这是一个图片示例2}
    \captionsetup{labelformat=english}
    \caption*{\bfseries Figure~2-1~This is a sample of image 2.}%需要手动修改
    \label{fig2}
\end{figure}

\subsection{示例Subsection-2}
\begin{table}[H]
	\centering
	\caption{这是一个表格示例2}
    \captionsetup{labelformat=english}
    \caption*{\bfseries Table~2-1~This is a sample of table 2.} %需要手动修改
	\begin{tabularx}{\textwidth}{>{\centering\arraybackslash}X>{\centering\arraybackslash}X>{\centering\arraybackslash}X}
			\toprule
			名称 & 名称 & 名称 \\ \midrule
			内容 & 4, 5, 6 & 123 \\
			内容 & 4, 5, 6 & 123 \\ \bottomrule
	\end{tabularx}
    \label{tab2}
\end{table}
%结论
\chapter*{结~~论}
这是一个结论示例。


% 附录
\include{appendix_new} % 阅读使用说明许可后,正式使用模板时移除此行代码。

% 参考文献
\bibliographystyle{template/bstutf8}
\clearpage
\phantomsection
\songti
\linespread{1.0}   
\centering
\addcontentsline{toc}{chapter}{参考文献} % 向目录中添加参考文献条目

\bibliography{references/main}

% 致谢
\chapter*{致~~谢}
\addcontentsline{toc}{chapter}{致谢} % 此命令添加不含序号的标题至目录
\raggedright
\setlength{\parindent}{2em} % 设置首行缩进为2em,根据需要调整缩进大小
感谢老师、感谢亲人、感谢朋友。
\end{document}
